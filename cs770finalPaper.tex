\documentclass{sigchi}

% Use this command to override the default ACM copyright statement (e.g. for preprints). 
% Consult the conference website for the camera-ready copyright statement.


%% EXAMPLE BEGIN -- HOW TO OVERRIDE THE DEFAULT COPYRIGHT STRIP -- (July 22, 2013 - Paul Baumann)
% \toappear{Permission to make digital or hard copies of all or part of this work for personal or classroom use is 	granted without fee provided that copies are not made or distributed for profit or commercial advantage and that copies bear this notice and the full citation on the first page. Copyrights for components of this work owned by others than ACM must be honored. Abstracting with credit is permitted. To copy otherwise, or republish, to post on servers or to redistribute to lists, requires prior specific permission and/or a fee. Request permissions from permissions@acm.org. \\
% {\emph{CHI'14}}, April 26--May 1, 2014, Toronto, Canada. \\
% Copyright \copyright~2014 ACM ISBN/14/04...\$15.00. \\
% DOI string from ACM form confirmation}
%% EXAMPLE END -- HOW TO OVERRIDE THE DEFAULT COPYRIGHT STRIP -- (July 22, 2013 - Paul Baumann)


% Arabic page numbers for submission. 
% Remove this line to eliminate page numbers for the camera ready copy
% \pagenumbering{arabic}


% Load basic packages
\usepackage{balance}  % to better equalize the last page
\usepackage{graphics} % for EPS, load graphicx instead
\usepackage{times}    % comment if you want LaTeX's default font
\usepackage{url}      % llt: nicely formatted URLs

% llt: Define a global style for URLs, rather that the default one
\makeatletter
\def\url@leostyle{%
  \@ifundefined{selectfont}{\def\UrlFont{\sf}}{\def\UrlFont{\small\bf\ttfamily}}}
\makeatother
\urlstyle{leo}


% To make various LaTeX processors do the right thing with page size.
\def\pprw{8.5in}
\def\pprh{11in}
\special{papersize=\pprw,\pprh}
\setlength{\paperwidth}{\pprw}
\setlength{\paperheight}{\pprh}
\setlength{\pdfpagewidth}{\pprw}
\setlength{\pdfpageheight}{\pprh}

% Make sure hyperref comes last of your loaded packages, 
% to give it a fighting chance of not being over-written, 
% since its job is to redefine many LaTeX commands.
\usepackage[pdftex]{hyperref}
\hypersetup{
pdftitle={Who to Blame? Understanding Distribution of Responsibility in Human-Robot Interaction},
pdfauthor={LaTeX},
pdfkeywords={SIGCHI, proceedings, archival format},
bookmarksnumbered,
pdfstartview={FitH},
colorlinks,
citecolor=black,
filecolor=black,
linkcolor=black,
urlcolor=black,
breaklinks=true,
}

% create a shortcut to typeset table headings
\newcommand\tabhead[1]{\small\textbf{#1}}


% End of preamble. Here it comes the document.
\begin{document}

\title{Who to Blame?\\ Understanding Distribution of Responsibility in Human-Robot Interaction}

\numberofauthors{3}
\author{
  \alignauthor  Stephen Gaschignard\\
    \affaddr{Affiliation}\\
    \affaddr{Address}\\
    \email{e-mail address}\\
  \alignauthor Mert Oguz\\
    \affaddr{Affiliation}\\
    \affaddr{Address}\\
    \email{e-mail address}\\    
  \alignauthor Xiang Zhi Tan\\
    \affaddr{Affiliation}\\
    \affaddr{Address}\\
    \email{e-mail address}\\
}

\maketitle

\begin{abstract}
Robotic products are not perfect and are prone to make mistakes. We conducted a study to understand how people attribute blame to different stakeholders when mistakes occur. We hypothesized that both the type and severity will influence the users? distribution of blame. We categorized the mistakes into psychological, physical and financial and separated them further by severity. Our first hypothesis states robot will receive more responsibility for psychological mistakes. Our second hypothesis states that the robot will receive more responsibility for less severe mistakes. We conducted a within-subject study where 49 participants watched six video scenarios with a robot making a mistake and attributed responsibility to each stakeholder. We found no evidence to support our hypotheses. The evidence indicated that the type of mistakes influenced the amount of blame, where severity of mistakes have no effect. Our study furthered the community understanding on how users might think about robotic mistakes.
\end{abstract}

\keywords{
	Guides; instructions; author's kit; conference publications;
	keywords should be separated by a semi-colon. \newline
	\textcolor{red}{Optional section to be included in your final version, 
  but strongly encouraged.}
}

\category{H.5.m.}{Information Interfaces and Presentation (e.g. HCI)}{Miscellaneous}

See: \url{http://www.acm.org/about/class/1998/}
for more information and the full list of ACM classifiers
and descriptors. \newline
\textcolor{red}{Optional section to be included in your final version, 
but strongly encouraged. On the submission page only the classifiers’ 
letter-number combination will need to be entered.}

\section{Introduction}

With the enhancement of robotics and artificial intelligence studies, robots have entered our lives in numerous ways. From industrial robots used in factories to everyday electronics like cleaning robots, humans take advantage of the cutting-edge technology that is designed for them. While studies about robotics deal with development and optimization of such technologies, the scope of human robot interaction studies expands through understanding psychological processes which humans come across while interacting with robots. Many notions which are vastly studied for human-human interaction such as trust, harm and perception of mistakes have been extended into human-robot interaction. \cite{freedy2007measurement}. It is clear that robots making mistakes during the interaction, therefore the performance of the robot is one of the most important reasons for trust to diminish. \cite{hancock2011meta}. Whether the other body is a human or robot, people tend to trust if there is consistency in the behavior of the other. Making mistakes on the other hand, is a natural outcome of poor analysis of the enclosing environment, sometimes caused by different deficiencies. Humans have a high try and fail learning capability, besides many creative mistake recovery strategies. With today's technology, robots are still not as talented as humans in terms of communication with other humans or direct interaction. Therefore, in the upcoming years, we predict that a number of studies will be conducted about the robot-made mistakes. As it is interesting to study why robots make mistake, studying how people perceive those mistakes and how they react in such situations is also crucial for designing successful mistake recovery systems.

Our research question focuses on the human perception of the possible robot mistakes in a restaurant setting. We have decided on this setting since we believe that robot waiters might be conventional in the near future. Restaurants are the places where at least a number of interactions occur between the customer and the waiter. These interactions include physical ones where the waiter brings the menu, dish or the bill. Conventionally, a waiter talks to a customer for getting the order or accepting the payment. These interactions have a psychological effect on the customer. Many customers decide on the amount of tip they wish to give based on the behavior of the waiter. Therefore, we categorized the interactions into different types to be studied.  We analyzed 3 different types of mistakes with 2 different severity levels. Our research is a within-participant study conducted with 49 participants (26 of them were females) where each participant is a U.S. resident. We hired participants through Amazon Mechanical Turk, paying them for completing our online questionnaire. In order to reach a more random participant population, we used video-based research instead of lab-based live settings. \cite{woods2006comparing} showed that video-based methods have reasonable results for new innovative studies. This approach enabled us to conduct our study with much smaller budget and time spending. Instead, we could focus on the scenarios and the quality of the questionnaire. 6 scenarios (3 types of mistakes: Physical, Psychological, Financial and 2 severity levels: Severe, Non severe) are designed for the study. After watching every video scenario, we asked participants how they attribute responsibility to each of the stakeholders (The robot, programmer of the robot, owner of the restaurant, manager of the restaurant, manufacturer). Then, we asked them what would be their reaction in such situations. We hypothesized that both the type and severity of the mistake will influence participants' distribution of blame. We further predicted that people will blame the robot more for psychological abuse. Our last hypothesis was people will blame the robot for non-severe situations, whereas they will blame other stakeholders for severe situations, if the mistake is a financial exploitation or physical abuse.

\section{Methodology}
\subsection{Research Question}
How do different types of errors in human-robot interaction affect the human’s perception on distribution of culpability?

\subsection{Hypotheses}
\textit{Hypothesis 1:} The participants will blame the robot regardless of the severity of the error if the error is a psychological abuse.

In the case of psychological harm, people tend to react immediately to the body who caused the harm. On the other hand, it is shown [1] that the people attribute a higher level of accountability to a robot than an ordinary computer system.

\textit{Hypothesis 2:} For the scenarios where the error is a financial exploitation or physical abuse, the participants will blame the robot for non-severe situations; whereas they will blame other stakeholders for severe situations.

The severe errors of physical or financial harm usually have long-term effects which makes people to blame the company or person that is responsible for the operation/ production of the robots. For instance when a plane accident occurs, people tend to blame the airline company even if the problem is caused by a rare mechanical malfunction. On the other hand, non-severe problems don’t require people to look for other stakeholders other than the robot itself.

\subsection{Experimental Design}
We used a 3 (type of mistake) x 2 (severity of mistake)  within-participants experimental design for this research study. Each participant viewed videos showing different scenarios with human-robot interactions - the videos represented each type and severity of harm that we were modifying to measure the effects of these factors.

\subsection{Experimental Task}
Each participant was introduced to a robot restaurant waiter through a video. After becoming familiar with the robot, the participant viewed a series of videos where the robot waiter harmed a human customer at the restaurant in some way. Each video represented each level of the factors described above:
\begin{enumerate}
\item Psychological Abuse
\begin{itemize}
\item Non-severe: robot offered a peanut-based dish to someone who had declared peanut allergies.
\item Severe: robot joked about a dead family member.
\end{itemize}
\item Financial Exploitation
\begin{itemize}
\item Non-severe: robot neglected to return change to the customer for the meal payment
\item Severe: robot claimed that a 50 dollar payment was only 10 dollars and requested extra payment from the customer
\end{itemize}
\item Physical Abuse
\begin{itemize}
\item Non-severe: robot dropped a small silverware item on the customer
\item Severe: robot dropped a hot food dish on the customer
\end{itemize}
\end{enumerate}

Each video was as realistic as possible - we used a real robot in the videos. We encouraged the participants to imagine that they were the human customer that was harmed in each video in order to elicit the most accurate reaction and perception of each events.

\subsection{Experimental Procedure}
The study was conducted over the internet where participants were asked to logged on to private website. After signing the online consent form in the website, the participant was first shown a one minute video commercial of the robot. This video showcase the intelligence and advance features in the robot. Afterwards, the participant was randomly assign the order of mistake types. In each mistake type, participant was randomly assign a severity condition and shown the video of mistake type with that severity condition. After finishing the video, the participant was asked to complete a questionnaire about their perception towards the robot in the video. Then, participant view the other condition video in the scenario. Upon completing all the conditions, the participant filled up a final questionnaires about demographic data. Once the participant finished the questionnaire, the participant was given between 0.35 and 0.50 dollars as compensation for their participation.

\subsection{Measurements}
In each condition, participants were asked to identify the possible parties responsible for the mistake in the questionnaire . The participants were given a list of parties involve such as the robot, manager on duty, company that owns the robot, programming team for the robot, company that owns the setting or nobody. In the list, participants were able to rank the parties they believed were responsible for the incident. If the participant believe there were no parties to be blamed, the participant could select nobody as their response. Besides identifying potential stakeholders, subjective measurements such as how angry are you, how angry do you feel the person in the video is, likeability of robot, intelligence of robot and independences of robot were also taken using a 1-7 likert scale.

To validate the successful manipulation of the independent variables, participants were also asked to rank the severity of each mistake on a 1-7 likert scale, where 1 is least severe and 7 as most severe. As the study was conducted through a web portal, additional validation questions that ask for specific details of the video such as color of robot were asked to ensure participants were following the procedure of the study.

\subsection{Participants}
50 participants (24 male and 26 female) were recruited through the Amazon mTruck Infrastructure and social media platform. One male participant's responses were removed from the data due to a failure to correctly respond to one of our data response validity questions. The participants were limited to people living in  United States.

\section{Findings}
\subsection{Manipulation Check}
We conducted a manipulation check to verify the success manipulation of our condition. After watching each video clip, we ask participants to rate the severity for each scenario. We found in all scenarios, the participants rated the severe condition significantly higher than the non-severe condition.

\begin{table}
  \centering
  \begin{tabular}{|c|c|c|c|}
    \hline
    Type & Non-Severe & Severe & p Value \\
    \hline
    Financial & 3.633 & 4.918 & 0.0003 \\
    \hline
    Psychological & 2.633 & 5.020 & $ < 0.0001$ \\
    \hline
    Psychological & 2.816 & 5.020 & $ < 0.0001$ \\
    \hline
  \end{tabular}
  \caption{Mean Difference between perceive severity}
  \label{tab:table1}
\end{table}

\subsection{Result}
To test our hypothesis, we carried analysis of variance (ANOVA) on the collected result. 

In our first hypothesis, we hypothesize that robots will be blamed significantly more than other stakeholders in the psychological harm condition. To verify the hypothesis we ran a one-way ANOVA on the result and found no evidence to support our hypothesize. We found participants blamed programmer the most(mean = 6.28) compare to other stakeholders. In relation to our hypothesis, participants blamed programmers($ F(1,582) =139.1, p < .0001$), owner ($ F(1,582) = 19.68, p < .0001$), manufacturer($F(1,582) = 38.28, p < .0001$) and manager($F(1,582) = 13.38, p = 0.0003$) more than the robot. Only customers were blamed significantly less than the robot($F(1,582) = 54.09, p < .0001$)

\begin{table}[h]
  \centering
  \begin{tabular}{|c|c|c|}
    \hline
    Stakeholders & Mean & Standard Deviation\\
    \hline
    Robot & 3.184 & 0\\
    \hline
    Customer & 1.255 & 0\\
    \hline
    Manager & 4.142 & 0 \\
    \hline
    Manufacturer & 4.806 & 0 \\
    \hline
    Owner & 4.34694 & 0 \\
    \hline
    Programmer & 6.286 & 0 \\
    \hline
  \end{tabular}
  \caption{Mean Difference between perceive severity}
  \label{tab:table2}
\end{table}

\section{Discussion}
Returning to our hypotheses, we were looking for our results to show that the participant would blame the robot more in situations involving psychological hare, and for the results to show more blame for other stakeholders in more severe situations. However, our results did not contribute to our hypotheses in any way. Based off the analyis of our data, there is no significant difference in how much the participant blames the robot for psychological harm versus other tpyes of harm. In addition to that, severity appears to have no effect on the distribution of blame whatsoever.

We did find that in all cases (between each level of severity and type of harm), the Programmer consistently took the highest amount of blame. This is interesting because it alludes to a higher level of understanding of how robot's work by the everyday person in the United States. At the basis of our study, we assumed that most participants would see robots as independent beings with their own independent actions. The high level of blame attributed to programmers implies that the everyday person understands that robots do not have their own free will, but that they are acting as instructed by those who programmed them. 

The data analysis also displayed a significant increase in the level of blame attributed to customers in scenarios with physical harm compared to psychological or financial harm. We see this as a result of our survey design and not necessrily as a result that can or should be applied to all scenarios with physical harm. In the two video scenarios that we use, the robot dropped an object (either silverware or a hot bowl) while handing over the object to the customer. It may have been possible for the customer to prevent the situation by catching the object before it fell, which may have been a reason to attribute more blame to the customer. Although the result is statistically significant, we do not see it as a result that is applicable outside of the context of our study.

There is another component of the survey design that may have influenced our results. When asking the participant to allocate blame between the different stakeholders, we listed six different stakeholders for the participant to choose from. We believe this influenced the participant to read through and fully consider each of the listed stakeholders instead of writing down who they instinctively blamed from their initial reaction to the scenarios. We may have seen a different distribution of blame between stakeholders using this approach compared to the data we collected.

\section{Conclusion}
Despite our initial hypotheses, a human would not attribute more blame to a robot for psychological harm or for less-severe mistakes. However, our results did indicate that there is a better understanding of robots by the general US population: we saw that people comprehend that robots are acting as instructed by programmers and not out of their own independent free will, and as such people will tend to blame programmers for a robot's mistakes.

\section{References format}
References must be the same font size as other body text.
% REFERENCES FORMAT
% References must be the same font size as other body text.
\bibliographystyle{acm-sigchi}
\bibliography{reference}
\end{document}
